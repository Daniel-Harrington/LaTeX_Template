% ===============================================
% Summer 2019
% hw_template.tex
% ===============================================

\documentclass{article}

\usepackage[margin=1.5in]{geometry} % Please keep the margins at 1.5 so that there is space for grader comments.
\usepackage{amsmath,amsthm,amssymb,hyperref}

\newcommand{\R}{\mathbf{R}}  
\newcommand{\Z}{\mathbf{Z}}
\newcommand{\N}{\mathbf{N}}
\newcommand{\Q}{\mathbf{Q}}

\newenvironment{theorem}[2][Theorem]{\begin{trivlist}
\item[\hskip \labelsep {\bfseries #1}\hskip \labelsep {\bfseries #2.}]}{\end{trivlist}}
\newenvironment{lemma}[2][Lemma]{\begin{trivlist}
\item[\hskip \labelsep {\bfseries #1}\hskip \labelsep {\bfseries #2.}]}{\end{trivlist}}
\newenvironment{claim}[2][Claim]{\begin{trivlist}
\item[\hskip \labelsep {\bfseries #1}\hskip \labelsep {\bfseries #2.}]}{\end{trivlist}}
\newenvironment{problem}[2][Problem]{\begin{trivlist}
\item[\hskip \labelsep {\bfseries #1}\hskip \labelsep {\bfseries #2.}]}{\end{trivlist}}
\newenvironment{proposition}[2][Proposition]{\begin{trivlist}
\item[\hskip \labelsep {\bfseries #1}\hskip \labelsep {\bfseries #2.}]}{\end{trivlist}}
\newenvironment{corollary}[2][Corollary]{\begin{trivlist}
\item[\hskip \labelsep {\bfseries #1}\hskip \labelsep {\bfseries #2.}]}{\end{trivlist}}

\newenvironment{solution}{\begin{proof}[Solution]}{\end{proof}}

\begin{document}

\large % please keep the text at this size for ease of reading.

% ------------------------------------------ %
%                 START HERE             %
% ------------------------------------------ %

{\Large Page 1 % Replace with appropriate page number 
\hfill MATA31H3 CSCA67H3 } %Remove the course(s) you are not writing the assignment for

\begin{center}
{\LargeName} % Replace with your name
\end{center}
\vspace{0.05in}
	
% -----------------------------------------------------
% The "enumerate" environment allows for automatic problem numbering.
% To make the number for the next problem, type " \item ". 
% To make sub-problems such as (a), (b), etc., use an "enumerate" within an "enumerate."
% -----------------------------------------------------

\begin{enumerate}

\item <Question>

\end{enumerate}

% ----------------------------------------------------

\end{document}